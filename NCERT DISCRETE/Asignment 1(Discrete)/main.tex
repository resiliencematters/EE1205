% \iffalse
\let\negmedspace\undefined
\let\negthickspace\undefined
\documentclass[journal,12pt,twocolumn]{IEEEtran}
\usepackage{cite}
\usepackage{amsmath,amssymb,amsfonts,amsthm}
\usepackage{algorithmic}
\usepackage{graphicx}
\usepackage{textcomp}
\usepackage{xcolor}
\usepackage{txfonts}
\usepackage{listings}
\usepackage{enumitem}
\usepackage{mathtools}
\usepackage{gensymb}
\usepackage{comment}
\usepackage[breaklinks=true]{hyperref}
\usepackage{tkz-euclide} 
\usepackage{listings}
\usepackage{gvv}                                        
\def\inputGnumericTable{}                                
\usepackage[latin1]{inputenc}                            
\usepackage{color}                                       
\usepackage{array}                                       
\usepackage{longtable}                                   
\usepackage{calc}                                        
\usepackage{multirow}                                    
\usepackage{hhline}                                      
\usepackage{ifthen}                                      
\usepackage{lscape}
\usepackage{amsmath}
\newtheorem{theorem}{Theorem}[section]
\newtheorem{problem}{Problem}
\newtheorem{proposition}{Proposition}[section]
\newtheorem{lemma}{Lemma}[section]
\newtheorem{corollary}[theorem]{Corollary}
\newtheorem{example}{Example}[section]
\newtheorem{definition}[problem]{Definition}
\newcommand{\BEQA}{\begin{eqnarray}}
\newcommand{\EEQA}{\end{eqnarray}}
\newcommand{\define}{\stackrel{\triangle}{=}}
\theoremstyle{remark}
\newtheorem{rem}{Remark}

\begin{document}

\bibliographystyle{IEEEtran}
\vspace{3cm}

\title{NCERT Mathematics Ex 5.3 Q8}
\author{EE23BTECH11059 - Tejas$^{}$% <-this % stops a space
}
\maketitle
\newpage

\Huge{QUESTION 8:}
\\ \\
        \medskip
        \Large
        Find the sum of first 51 terms of an AP whose second and third terms are 14 and 18
respectively.
    \bigskip
    \Large
    \\
        Given : a\textsubscript{2}=14, a\textsubscript{3}=18
        \begin{align}
            a+d&=14 \\
            a+2d&=18 
        \end{align}
        
        After solving (1) and (2) \\
           \begin{align}
             d=4\\ a=10
        \end{align}  

        To find the sum till 51 terms: \\
        \begin{align}
            S\textsubscript{51} &= \frac{51}{2} [(2)(4)+(50)(4)] \notag \\
            S\textsubscript{51} &= \frac{51}{2} [208] \notag 
        \end{align}
        \begin{equation}
    \boxed{S\textsubscript{51} &= 5304}
\end{equation}













\renewcommand{\thefigure}{\theenumi}
\renewcommand{\thetable}{\theenumi}





\end{document}
