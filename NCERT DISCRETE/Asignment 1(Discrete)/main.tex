% \iffalse
\let\negmedspace\undefined
\let\negthickspace\undefined
\documentclass[journal,12pt,twocolumn]{IEEEtran}
\usepackage{cite}
\usepackage{amsmath,amssymb,amsfonts,amsthm}
\usepackage{algorithmic}
\usepackage{graphicx}
\usepackage{textcomp}
\usepackage{xcolor}
\usepackage{txfonts}
\usepackage{listings}
\usepackage{enumitem}
\usepackage{mathtools}
\usepackage{gensymb}
\usepackage{comment}
\usepackage[breaklinks=true]{hyperref}
\usepackage{tkz-euclide} 
\usepackage{listings}
\usepackage{gvv}                                        
\def\inputGnumericTable{}                                
\usepackage[latin1]{inputenc}                            
\usepackage{color}                                       
\usepackage{array}                                       
\usepackage{longtable}                                   
\usepackage{calc}                                        
\usepackage{multirow}                                    
\usepackage{hhline}                                      
\usepackage{ifthen}                                      
\usepackage{lscape}
\usepackage{amsmath}
\newtheorem{theorem}{Theorem}[section]
\newtheorem{problem}{Problem}
\newtheorem{proposition}{Proposition}[section]
\newtheorem{lemma}{Lemma}[section]
\newtheorem{corollary}[theorem]{Corollary}
\newtheorem{example}{Example}[section]
\newtheorem{definition}[problem]{Definition}
\newcommand{\BEQA}{\begin{eqnarray}}
\newcommand{\EEQA}{\end{eqnarray}}
\newcommand{\define}{\stackrel{\triangle}{=}}
\theoremstyle{remark}
\newtheorem{rem}{Remark}

\begin{document}

\bibliographystyle{IEEEtran}
\vspace{3cm}

\title{NCERT Mathematics Ex 9.4 Q6}
\author{EE23BTECH11059 - Tejas$^{}$% <-this % stops a space
}
\maketitle
\newpage

        \Large
        1) Find the sum to n terms of\\$3 \times 8 + 6 \times 11 + 9 \times 14 + ...$
    \vspace{1cm}
    \Large
    \\
        \Large
        SOLUTION: \\
        \Large
        Writing the series in terms of\\summation:
        \begin{align}
            &\sum_{r=0}^{n}3r\times(5+3r) \notag \\
            &\sum_{r=0}^{n}15r+9r^2 
        \end{align}
        Using formulas for the sum of n terms (i) and sum of the squares of the n terms (ii) \\
          \begin{align}
            \sum_{r=0}^{n}r&=\frac{n(n+1)}{2} \hspace{2cm}(i)\notag \\
            \sum_{r=0}^{n}r^2&=\frac{n(n+1)(2n+1)}{6} \hspace{2cm}(ii)\notag 
        \end{align}
        
        
        Equation (1) evaluates to
             \begin{align*}
                \frac{15n(n+1)}{2} + \frac{9n(n+1)(2n+1)}{6} 
            \end{align*}
            
            \begin{flushleft}
                \hspace{0.5cm} \notag
                \boxed{\frac{3n(n+1)(2n+6)}{2}} \notag
            \end{flushleft}
             
             
             
        

        













\renewcommand{\thefigure}{\theenumi}
\renewcommand{\thetable}{\theenumi}





\end{document}
